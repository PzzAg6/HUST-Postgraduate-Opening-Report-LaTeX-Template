%% Local Variables:
%%% mode: latex
%%% TeX-master: t
%%% End:

\documentclass[draftformat,mathCMR]{HUSTthesis}
%\documentclass[finalformat,mathCMR]{HUSTthesis}
% 所有其它可能用到的包都统一放到这里了,可以根据自己的实际添加或者删除。这样做
% 主要是为了避免class文件过于臃肿。
\usepackage{HUSTtils}
% \setmainfont{Times New Roman}[Scale=.9]
\setmainfont{Times New Roman}

%\includeonly{body/chap02}
% \setcitestyle{authoryear, open={( },close={ )}}

\usepackage[stable]{footmisc}
\usepackage{xcolor}
\usepackage{soul}


\newcommand*\red[1]{\textcolor{red}{#1}}
% \setul{0.5ex}{0.3ex}
\usepackage{xeCJKfntef}
\newcommand\redline[1]{\CJKunderline[skip=false,format=\color{red}]{#1}}

\usepackage{mathtools}
% \let\proof\relax
% \let\endproof\relax
% \let\theoremstyle\relax
% \let\newtheoremstyle\relax
% \usepackage{amsthm}
% \usepackage{amssymb}
\usepackage{amsmath,amssymb,bm}
\DeclareMathOperator*{\argmax}{argmax}
\DeclareMathOperator*{\argmin}{argmin}
\DeclareMathOperator{\trans}{{\mathrm{T}}}
\DeclareMathOperator*{\avg}{Avg}
\newcommand*\abs[1]{\left \lvert#1 \right \rvert}
\newcommand*\norm[1]{\left \lVert #1 \right \rVert}
\newcommand\x{\bm{x}}
\newcommand\y{\bm{y}}
\newcommand\e{\bm{e}}
\newcommand\h{\bm{h}}
\newcommand\w{\bm{w}}
\newcommand\g{\bm{g}}
\newcommand\m{\bm{m}}
\newcommand\bs{\bm{s}}
\newcommand\bp{\bm{p}}
\newcommand\brho{\bm{\rho}}
\newcommand\bv{\bm{v}}
\newcommand\bu{\bm{u}}
\newcommand\mathr{\mathbb{R}}
\newcommand\mathd{\mathbb{D}}

% \usepackage[e]{esvect}
% \newcommand{\uv}[2][2mu]{\vec{#2\mkern-#1}\mkern#1}

% *** SPECIALIZED LIST PACKAGES ***
\usepackage{algorithm, algorithmic}
% \newcommand{\algorithmautorefname}{Algorithm}

\usepackage{breqn}

\usepackage{pifont}
\newcommand{\cmark}{\ding{51}}
\newcommand{\xmark}{\ding{55}}
\newcommand{\rfist}{\ding{192}~}
\newcommand{\rsecond}{\ding{193}~}
\newcommand{\rthird}{\ding{194}~}

\usepackage{marginnote}
\usepackage{multirow,makecell}
\usepackage{tabularx}
% \usepackage{ltablex} %load to using long tabularx
% \keepXColumns %solve the problem that tabularx does not fill full textwidth when using ltablex
\newcolumntype{Y}{>{\centering\arraybackslash}X}
\newcolumntype{P}[1]{>{\centering\arraybackslash}p{#1}}

\newcommand\ph{$\phantom{1}$} 
\newcommand\s{$^\star$}

\newlength{\twosubht}
\newsavebox{\twosubbox}

% 修改autoref前后间距,解决标签应用后的前后间距问题
\usepackage{letltxmacro}
\LetLtxMacro\oldautoref\autoref
\DeclareRobustCommand{\autoref}[1]{{\let\nobreakspace\space\oldautoref{#1}\space}}

\newcommand*\secref[1]{\ref{#1}节}

\usepackage{cleveref}
\newcommand{\crefrangeconjunction}{~--~}
\Crefname{table}{表}{表}
\Crefname{figure}{图}{图}

% %%  using line number.
% \usepackage[pagewise]{lineno}
% \makeatletter
% \def\makeLineNumberLeft{%
%   \linenumberfont\llap{\hb@xt@\linenumberwidth{\LineNumber\hss}\hskip\linenumbersep}% left line number
%   \hskip\columnwidth% skip over column of text
%   \rlap{\hskip\linenumbersep\hb@xt@\linenumberwidth{\hss\LineNumber}}\hss}% right line number
% \leftlinenumbers% Re-issue [left] option
% \makeatother

\addbibresource[location=local]{ref/thesis.bib}%使用biblatex


\begin{document}
%定义所有的eps文件在 figures 子目录下
\graphicspath{{figures/}}

% 生成封面,版权页,摘要

\frontmatter

%%% Local Variables:
%%% mode: Xelatex
%%% TeX-master: t
%%% End:

\ctitle{开题报告题目\\第二行}

\xuehao{XX}
\csubjectname{XX} \cauthorname{XX}
\cdepartment{XX学院}
\csupervisorname{XX} \csupervisortitle{教授}



\makecover
\makeattention

%目录
\pagenumbering{gobble}
\tableofcontents

% 对照表
% \include{body/denotation}

\mainmatter

\raggedbottom
%% using line count
% \linenumbers

% \nolinenumbers
%%% 结论
%\input{body/chapter/cnn.tex}
%\input{body/chapter/univ.tex}

\chapter{选题来源、背景、目的和意义}

\section{选题来源}
引用方式:\begin{description}
	\item[句首引用] \verb|\citet{alitappehMultirobotExplorationTask2022}|。\citet{alitappehMultirobotExplorationTask2022}
	\item[句中引用] \verb|\cite{alitappehMultirobotExplorationTask2022}|。\cite{alitappehMultirobotExplorationTask2022}
\end{description}

带圈标号:\begin{description}
	\item[代码] \verb|\ding{192}~\ding{200}|
	\item[显示] \ding{192}\~{} \ding{200}
\end{description}


一个三线表
表~\ref{tab:template-files} 列举了本模板主要文件及其功
能。
\begin{table}[htb]
	\centering
	\caption{模板文件清单}
	\label{tab:template-files}
	\begin{minipage}[t]{0.8\linewidth} % 如果想在表格中使用脚注,minipage是个不错的办法
		\begin{tabular*}{\linewidth}{m{3cm}m{10cm}}
			\toprule[1.5pt]
			{\hei 文件名}  & {\hei 描述} \\\midrule[1pt]
			HUSTthesis.cls & 模板类文件\\
			HUSTbib.bst    & 参考文献~Bib\TeX{} 样式文件\\
			HUSTtils.sty   & 常用的包和命令写在这里,减轻主文件的负担\\
			\bottomrule[1.5pt]
		\end{tabular*}
	\end{minipage}
\end{table}



\section{研究背景}
\section{研究目的和意义}



\chapter{国内外研究现状}



\section{文献小结与评述}


\chapter{研究方案}
\section{研究框架和主要内容}


%\chapter{研究计划与进度安排}

\section{研究方法和技术路线}


换页的地方间距不够可以插入一个换行

\section{论文整体结构}
\def\labelenumi{\textbf{\theenumi} }\def\theenumi{\arabic{enumi}}
\def\labelenumii{\theenumi.\theenumii }\def\theenumii{\arabic{enumii}}
%\begin{mdframed}
\begin{mdframed}[everyline=true]\label{enum}
	\begin{enumerate}
		\item \textbf{绪论}
		\begin{enumerate}
			\item 背景研究
			\item 研究目的与意义
			\item 研究内容与章节结构
			\item 研究方法与技术路线
			\item 主要创新点
		\end{enumerate}
		\item \textbf{国内外相关研究综述}
		\begin{enumerate}
			\item 相关研究1
			\item 相关研究2
			\item 相关研究3\\%换行
		
			\item 文献小结与评述
		\end{enumerate}
		\item \textbf{问题1}
		\begin{enumerate}
			\item 问题提出
			\item 问题描述与数学模型
			\item 算法实现
			\item 实验结果与分析
			\item 本章小结
		\end{enumerate}
		\item \textbf{问题2}
		\begin{enumerate}
			\item 研究背景
			\item 问题描述与数学模型
			\item 算法实现
			\item 实验结果与分析
			\item 本章小结
		\end{enumerate}
		\item \textbf{全文总结及研究展望}
		\begin{enumerate}
			\item 全文总结
			\item 研究展望
		\end{enumerate}
	\end{enumerate}
\end{mdframed}

\chapter{研究进度安排}





\phantomsection
\addcontentsline{toc}{chapter}{参考文献}
\printbibliography[heading=bibliography,title=参考文献]

\makesignature

% ---------------------------------------------

%%% 附录
%\begin{aendix}
%  \input{body/aendix/a.tex}
%
%\end{aendix}

\end{document}
